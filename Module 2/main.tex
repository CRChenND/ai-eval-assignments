\documentclass{article}
\usepackage{graphicx} % Required for inserting images

\title{AI eval M2 assignment}
\author{Chaoran Chen}
\date{February 2026}

\begin{document}

\maketitle

\section{Introduction}

Large language models (LLMs) are increasingly deployed in high-stakes decision-making and engineering planning tasks. Despite strong reasoning and language capabilities, LLMs remain susceptible to hallucinations—confidently generated but factually incorrect or scientifically implausible outputs. Hallucinations become particularly problematic in multi-stage, cross-domain engineering problems, where errors may propagate across interdependent constraints and produce technically infeasible system designs.

This assignment investigates how different inference strategies influence hallucination rates and factual reliability in LLM-generated engineering plans. Specifically, we evaluate model performance under a hypothetical yet scientifically constrained deployment scenario: operating an 8-GPU large language model server on the Martian surface. The scenario intentionally integrates multiple technical domains, including thermodynamics, power systems engineering, orbital logistics, and distributed software deployment, thereby creating a realistic stress test for model reliability.

Rather than evaluating solution correctness alone, this study focuses on auditing model reasoning failures across inference strategies. We compare five distinct inference approaches, including zero-shot prompting, chain-of-thought reasoning, prompt iteration, reasoning-specialized models, and tool-augmented agentic inference. We then conduct a systematic factual audit by comparing generated outputs against established scientific and engineering constraints.

The contributions of this study are threefold:

\begin{itemize}
\item A structured evaluation framework for measuring hallucinations in multi-domain engineering planning tasks.
\item A comparative analysis of inference strategies and their impact on factual integrity.
\item A scientifically grounded deployment plan derived from the most reliable inference strategy.
\end{itemize}


\section{Deployment Scenario and Task Definition}

We consider a hypothetical deployment scenario in which a large language model must be operated locally on a high-performance computing system located on the Martian equatorial surface. The system consists of an 8-GPU NVIDIA server installed in a standard rack configuration.

Participants are tasked with generating a technical deployment plan that addresses the following stages:

\begin{itemize}
\item Transportation and landing of the hardware infrastructure on Mars.
\item Power system design capable of sustaining continuous server operation.
\item Thermal management and environmental adaptation within the Martian atmosphere.
\item Software deployment and operational orchestration of the LLM infrastructure.
\end{itemize}

This scenario requires coordination across several engineering domains, including power electronics, thermal physics, planetary environmental science, and distributed computing systems. The complexity of cross-domain constraints makes this task particularly suitable for evaluating hallucination behaviors in LLM-generated outputs.

The primary objective is not to produce a perfect engineering solution but to evaluate the reliability of model-generated reasoning under different inference strategies.


\section{Inference Strategy Comparison}

To evaluate the impact of inference design on model reliability, we compare five inference strategies that represent distinct prompting and reasoning paradigms.

\subsection{Zero-Shot Prompting}

Zero-shot prompting involves directly querying the model with the deployment task without providing structured reasoning guidance. This baseline captures default model behavior under minimal prompt engineering.

\subsection{Chain-of-Thought Reasoning}

Chain-of-thought (CoT) prompting encourages the model to explicitly decompose the problem into intermediate reasoning steps. Prior work suggests that CoT prompting can improve logical consistency in complex problem-solving tasks.

\subsection{Prompt Iteration}

Prompt iteration involves repeated or refined prompt queries to assess the stability and variance of model outputs. This approach evaluates whether hallucination patterns persist across multiple generations.

\subsection{Reasoning-Optimized Models}

We evaluate specialized reasoning-oriented model variants designed for complex analytical tasks. These models often incorporate architectural or training modifications intended to enhance multi-step reasoning.

\subsection{Agentic Tool-Augmented Inference}

Agentic inference integrates external knowledge retrieval or tool-calling mechanisms. This approach allows the model to verify environmental data, such as Martian solar irradiance or atmospheric conditions, thereby potentially reducing hallucinations.

\section{Evaluation Methodology}

\subsection{Factuality Auditing}

We perform a systematic factual audit by cross-referencing model-generated technical claims with established scientific datasets and engineering references. Evaluation focuses on three primary constraint categories:

\textbf{Power Feasibility:} Assessment of whether proposed energy sources can realistically sustain 8-GPU server operation under Martian environmental conditions.

\textbf{Environmental Physics:} Evaluation of thermal management strategies and their compatibility with Martian atmospheric properties.

\textbf{Deployment Validity:} Verification of software deployment workflows and system orchestration feasibility.

\subsection{Hallucination Detection}

Hallucinations are identified when generated outputs contain scientifically unsupported, physically implausible, or internally inconsistent claims.

\subsection{Quantitative Metrics}

We evaluate inference strategies using:

\begin{itemize}
\item Hallucination rate (percentage of incorrect technical claims)
\item Constraint violation count
\item Internal logical consistency score
\item Cross-run output variance
\end{itemize}

\section{Hallucination Taxonomy}

To systematically characterize reasoning failures, we categorize hallucinations into four types:

\textbf{Scientific Hallucinations:} Violations of known physical or engineering laws.

\textbf{Numerical Hallucinations:} Incorrect quantitative estimates such as power consumption or thermal load.

\textbf{Procedural Hallucinations:} Invalid or incomplete deployment workflows.

\textbf{Cross-Domain Inconsistencies:} Conflicts between constraints across engineering domains.

\end{document}
